% -*-LaTeX-*-
% Converted automatically from troff to LaTeX
% by tr2latex 2.5
% on Fri May  4 17:23:46 2018
% tr2latex was written by Kamal Al-Yahya at Stanford University <Kamal%Hanauma@SU-SCORE.ARPA>
% and substantially enhanced by Christian Engel at RWTH Aachen <krischan@informatik.rwth-aachen.de>
% it is currently maintained by Dirk Jagdmann <doj@cubic.org>
% visit http://www.ctan.org/pkg/tr2latex for details.
%
% troff input files: minecrafty.6
% ECG: we'll see if the conversion is invertable by latex2man

\documentclass[]{article}

\usepackage{troffman}
\def\mcy{{\bf minecrafty} }
\begin{document}
%--------------------------------------------------
% start of input file: minecrafty.6
%
% Copyright (c) 2018 brickviking (brickviking@gmail.com),
%     Thu Apr 12 17:05:00 NZST 2018
%
% Permission is granted to copy, distribute and/or modify this document
% under the terms of the GNU Free Documentation License, Version 1.3
% or any later version published by the Free Software Foundation;
% with no Invariant Sections, no Front-Cover Texts, and no Back-Cover Texts.
% A copy of the license is included in the separate document entitled "GNU
% Free Documentation License".
%
% This manual is distributed in the hope that it will be useful,
% but WITHOUT ANY WARRANTY; without even the implied warranty of
% MERCHANTABILITY or FITNESS FOR A PARTICULAR PURPOSE.  See the
% GNU General Public License for more details.
%
% You should have received a copy of the GNU General Public
% License along with this manual; if not, download it from the www.fsf.org
% website or write to the Free Software Foundation, Inc., 59 Temple Place,
% Suite 330, Boston, MA 02111, USA.
%
% Version 0.1.3 for minecrafty 0.62.
% TODO: fix the Aether reference, remove all the obsolete entries
%
% \font\tenrm=universalis at 10pt
\phead{MINECRAFTY}{6}{2018-04-12}{Linux}{Java games}
\shead{NAME}
minecrafty -- A Minecraft launcher front end.
\shead{SYNOPSIS}
\begin{TPlist}{}
  \item[{\mcy\ 
  \lbrack{\bf -hvacTSBEOP}\rbrack\ 
  \lbrack{\bf -b\ {\it n.n}}\rbrack\ 
  \lbrack{\bf -g\ {\it n.n}}\rbrack\ 
  \lbrack{\bf -m \ {\it 0$|$1$|$2}}\rbrack\ 
  \lbrack{\bf -r\ {\it nn.n}}\rbrack\ 
  \lbrack{\bf -z\ {\it XxY}\rbrack} ...}] % end of first item
  \item[{\mcy\lbrack{\bf -f\ {\it launcher\_name$|$help}}\rbrack\ ...}]
  \item[{\mcy\lbrack{\bf -l\ {\it launcher\_name$|$help}}\rbrack\ ...}]
  \item[{\mcy\lbrack{\bf -s\ {\it servername}}\rbrack\ ...}]
  \item[{\mcy ...}]
\end{TPlist}

\shead{DESCRIPTION}
{\bf minecrafty}
executes a java launcher for the computer game Minecraft (Java Edition).
Optionally, you can download or start Minecraft launchers or download
java clients and servers for vanilla Minecraft. 

When
{\bf minecrafty}
is started with no arguments, it runs the Minecraft launcher found in\break
{\it \$\{HOME\}/.minecraft/bin/minecraft\_launcher.jar}
if you have already downloaded and installed it in that location.

It is
{\it not}
for the Bedrock edition of Minecraft; that already has its own launcher
and only works under Windows 10.
\shead{REQUIREMENTS}
Most of these requirements actually relate to the ability to run Minecraft,
not specifically the minecrafty front end.
\sshead{Hardware}
\begin{TPlist}{}
\item[{{\bf Memory}}]\nwl
Your machine should have an absolute minimum of 4 Gb to play Minecraft.
It should ideally have more memory than this, especially for recent FTB
modpacks and other modpack collections.
\item[{{\bf 3-D Video card}}]\nwl
You should have a recent PCI Express 3-D video card that can play Minecraft, preferably
with 512 Mb of video memory or more. Laptop video and integrated video
chipsets may or may not work well for Minecraft, as they are often slower
or may have less memory for video. AGP video cards aren't powerful enough,
and are far too old.
\item[{{\bf Intel or AMD CPU}}]\nwl
This should be from 2008 or later, as earlier CPUs simply won't run
Minecraft well, if at all. Anything that's a Core/2 Quad or later works,
but an i3/i5/i7 or an AMD equivalent would be better. A 32-bit CPU can run
Minecraft, but to get the best performance, your CPU and Linux should be
running in 64-bit mode.
\end{TPlist}
\sshead{Software}
\begin{TPlist}{}
\item[{{\bf Minecraft}}]\nwl
A paid account for Minecraft. The unpaid version won't work, or at best
you'll have a demo that lasts you 100 minutes (about five in-game days).
You can purchase Minecraft from the
{\it minecraft.net}
website. Some local stores may sell a Minecraft prepaid card with a code
you can redeem at 
{\it minecraft.net.}
\item[{{\bf java}}]\nwl
A recent version of java 8, either JavaSE or SDK will do. OpenJDK will also do fine.
If Minecraft will run using your java, you're good. 
\item[{{\bf Linux}}]\nwl
Any recent up-to-date 64-bit version of Linux later than 2014 will do, I've tested
this under Ubuntu and Fedora, both work fine. I cannot confirm if this script
or its dependencies will even work under Cygwin, let alone wine.
Minecraft itself works on 32-bit versions of Linux up to 1.12.2, but not from 1.13 onwards, as lwjgl has dropped all support for 32-bit Linux in 2016.
\item[{{\bf Xorg}}]\nwl
Linux running in graphical (Xorg) mode on a local screen, in 3-D accelerated
mode. Your desktop can be GDM, KDM, or whatever. Wayland also works,
as long as it provides the same features.
\item[{{\bf xrandr}}]\nwl
This is available in all common Linux distributions.
\item[{{\bf bash4}}]\nwl
You need bash installed, preferably bash4 or later. I don't think this
script will work under dash or most other shells.
\item[{{\bf awk}}]\nwl
The gawk package should provide awk, and should be available in all Linux
distributions.
\item[{{\bf wget}}]\nwl
This does the fetching of launchers and client jar files. I haven't tested
this with curl or other web fetching programs.
\item[{{\bf terminal}}]\nwl
Not an absolute requirement, but is a good idea so you can see output.
Terminal examples include Gnome Terminal, Konsole, or even xterm.
\end{TPlist}
\shead{OPTIONS}
\sshead{General options}
Options for minecrafty are as follows:
\begin{TPlist}{}
\item[{{\bf -h}}]
prints the main help screen for 
{\bf minecrafty}{\rm .}
You can also use
{\bf -?}
to view the same help.
Further help details are described under the
{\bf Launcher options}
and
{\bf Fetching options}
sections.
\item[{{\bf -v}}]
show minecrafty's version string
\item[{{\bf -a}}]
appends launcher output to
{\it \$\{HOME\}/.minecraft/minecrafty.output.txt,}
creating it if it doesn't exist. You should probably
trim this file from time to time, or use the
{\bf -c}
switch instead.
\item[{{\bf -c}}]
creates 
{\it \$\{HOME\}/.minecraft/minecrafty.output.txt}
for logging launcher output, emptying the file first if it already exists.
\end{TPlist}
\sshead{Memory options}
These options affect how much memory the launcher will start with. This does
{\it not}
affect how much memory that Minecraft itself will start with, just the amount
assigned to the launcher. Historically (previous to 1.5.2) the amount assigned
on the commandline affected how much memory the Minecraft client started with.
In addition, some launchers (the ones that don't use Java to start themselves)
don't require a specific memory setting, so these options are ignored for them.
\begin{TPlist}{}
\item[{{\bf -T}}]
runs launcher with a tiny amount of memory (256 mb).
\item[{{\bf -S}}]
runs launcher with a normal amount of memory (512 Mb).
This is the default if no memory option is chosen.
\item[{{\bf -B}}]
runs launcher with a larger amount of memory (1024 mb).
\item[{{\bf -E}}]
runs launcher with heaps of memory, only on 64-bit systems (2 Gb).
\end{TPlist}
\sshead{Audio options}
These are needed if you need pulseaudio to be tweaked or simply get out of
the way.
\begin{TPlist}{}
\item[{{\bf -O}}]
create OSS-like pulseaudio device.
\item[{{\bf -P}}]
suspends pulseaudio while Minecraft runs.
\end{TPlist}
\sshead{Graphical options}
These affect what you see on your screen.
\begin{TPlist}{}
\item[{{\bf -b}{\it \ n.n}
}]\nwl
set the brightness to the value specified as 
{\it n.n,}
for example 
{\bf -b 1.0.}
If unset, 1.0 is used. 0.7 darker, 1.2 lighter.
\item[{{\bf -g}{\it \ n.n}
}]\nwl
set the gamma offset. If unset, 1.0 is used. 0.7 lighter, 1.2 darker.
\item[{{\bf -m}{\it \ $<$0$|$1$|$2$|$3$>$}
}]\nwl
select monitor to apply gamma correction to.
\item[{{\bf -r}{\it \ nn.n}
}]\nwl
set a valid refresh rate for the second monitor, as Minecraft has a habit
of setting this to the recommended value (often 60 Hz) instead of the
value used when your screen started. This applies to people who have
specifically set 75 Hz or some other refresh rate.
\item[{{\bf -z}{\it \ XxY}
}]\nwl
set required resolution (1920x1080 default).
\end{TPlist}
\sshead{Fetching options}
\begin{TPlist}{}
\item[{{\bf -f help}}]\nwl
produces help for fetching launchers and Minecraft versions.
\item[{{\bf -f\ $<$}{\it n.n}{\bf \ $|$}{\it n.n.n}{\bf $>$}
}]
retrieves a release version of the
{\it minecraft.jar}
client, examples of this are
{\bf 1.0}
or
{\bf 1.10.2}{\rm .}
This script doesn't fetch versions earlier than 1.0 or pre-release client
versions, but the default launcher will now fetch earlier versions such as
alpha and beta.
\item[{{\bf -f $<$YEARwWEEKx$>$}}]\nwl
retrieves a snapshot version of minecraft.jar, such as 17w49b.
{\bf Year}
starts from 11, and (so far) goes to 18 but will possibly range up to
29. 
{\bf WEEK}
ranges from 01 to 53,
{\bf x}
is a letter from a to m. The earliest snapshot version that exists in this
format is 11w47a; although there were earlier snapshot versions, they don't
follow this naming convention.
\item[{{\bf -f}{\it \ launcher\_name}
}]\nwl
retrieves
{\it launcher\_name}
from the relevant remote host such as the FTB website.
The valid 
{\it launcher\_name}
choices are:
% \ind{1\parindent}
% TODO: turn this into a two-column list without border
% |name|meaning|
{\begin{description}
\item[{\bf launcher}]{}
\item[{\bf default}]{}
\item[{\bf vanilla}]{\nwl
downloads the default launcher.}
\item[{\bf atl}]{\nwl
download AT Launcher for Linux.}
\item[{\bf ftb}]{\nwl
download FTB launcher.}
\item[{\bf magic}]{\nwl
download MagicLauncher.}
\item[{\bf multi}]{\nwl
download MultiMC, this requires a recent version of Linux and glibc.}
\item[{\bf technic}]{\nwl
download Technic launcher.}
\item[{\bf terra}]{\nwl
download TerraFirmaCraft launcher (1.7.10 only).}
  \end{description}
} % Inner list
% ### Obsoleted entries ###
% .B aether
% download Aether Launcher. This seems to be the latest Aether II project.
% Currently the launcher doesn't work, and development versions of 1.10.2 and
% 1.11.2 use a profile created on the default launcher.
% .TP
% .B bukkitrb
% download latest recommended bukkit build (obsolete due to a DMCA claim)
% .TP
% .B craft
% download Craftland installer for latest Aether prerelease.
% .TP
% .B digiex
% download DigiEX launcher (obsolete, may not work).
% .TP
% .B mvc
% download Minecraft Version Changer from TunkDesign. (probably obsolete)
% .TP
% .B skmc
% download development version of SKMCLauncher.jar (obsolete, may not work).
% .TP
% \ind{1\parindent}
\end{TPlist} % Outer list
\sshead{Launcher options}
\begin{TPlist}{}
\item[{{\bf -l help}}]
produces help for starting Minecraft launchers.
\item[{{\bf -l}{\it \ launcher\_name}
}]
starts specified launcher. So does simply typing
{\bf minecrafty}
at the prompt. The valid
{\it launcher\_name}
choices are:
\ind{1\parindent}
% TODO: turn this into a two-column list without border
% |name|meaning|
    {\begin{description}
      \item[{\bf default}]{}
      \item[{\bf launcher}]{}
      \item[{\bf vanilla}]{\nwl run default launcher}
      \item[{{\bf atl}}]{\nwl run AT Launcher}
      \item[{{\bf ftb}}]{\nwl run FTB launcher}
      \item[{{\bf magic}}]{\nwl run MagicLauncher}
      \item[{{\bf multi}}]{\nwl run MultiMC}
      \item[{{\bf technic}}]{\nwl run Technic launcher}
      \item[{{\bf terra}}]{\nwl run TerraFirmaCraft launcher (1.7.10 only)}
      \end{description}
    }
% ### Obsoleted entries ###
% .TP
% .B aether
% run Aether Launcher
% .TP
% .B craft
% run CraftLand launcher
% .TP
% .B digiex
% run DigiEX launcher
% .TP
% .B mvc
% run Minecraft Version Changer (MVC.jar)
% .TP
% .B skmc
% run development version of SKMCLauncher.jar
% \ind{1\parindent}
\end{TPlist}
\sshead{Starting a server}
\begin{TPlist}{{\bf -s}{\it \ server\_name.jar}
}
\item[{{\bf -s}{\it \ server\_name.jar}
}]\nwl starts {\it \$\{HOME\}/.minecraft/server/server\_name.jar ,}
you will need to download the server of your choice into this location.

\end{TPlist}\shead{EXAMPLES}
\begin{TPlist}{{\bf minecrafty}}
\item[{{\bf minecrafty}}]\nwl executes the default launcher with the default amount of memory assigned to it.
\item[{{\bf minecrafty -l ftb}}]\nwl executes the FTB launcher with the default amount of memory assigned to it.
\item[{{\bf minecrafty -f 1.12.2}}]\nwl fetch the jar file for the 1.12.2 client, saving it in the correct place.
\item[{{\bf minecrafty -f 18w15a}}]\nwl fetch the jar file for the 18w15a snapshot (development version) client file.
\item[{{\bf minecrafty -m 1 -g 1.3 -z 1360x768 -r 75 -c}}]\nwl execute the default launcher, setting the screensize, refresh rate and gamma
for the second monitor. Console output from the launcher will be in
{\it \$\{HOME\}/.minecraft/minecrafty.output.txt}
for you to look at if you have problems later. The user's probably going to
have to put the minecraft window on the second monitor and deal with all that
lovely desktop manager jazz for themselves.
\item[{{\bf minecrafty -s spigot-1.12.2.jar}}]\nwl This runs the spigot server file. You'll need to get a server jar file
and put it into
{\it \$\{HOME\}/.minecraft/server/}

\end{TPlist}\shead{BUGS}
No doubt there are plenty, I still haven't winkled out all the bugs in the
graphics options yet, and most of the launchers that I used to support
downloads for aren't around any more. Thankfully, I can recommend the FTB
launcher, which is kept up to date.

This seems to take no notice of an existing MCBase or Minecraft home where
that differs from the normal user's home directory.

The return value from minecrafty will be different if you use the logfile
option
{\bf -a}
or
{\bf -c}
as the exit value will actually be from the program used to capture the output.

\shead{AUTHORS}
Me, of course (brickviking@gmail.com). I've had a little bit of help from others too.

\shead{ACKNOWLEDGEMENTS}
\begin{TPlist}{{\bf Mojang}}
\item[{{\bf Mojang}}]\nwl for the game that got seriously popular. Yes, Infiniminer got there first, but Mojang made Minecraft what it is today.
\item[{{\bf The writers of Java}}]\nwl If it wasn't for Java, then Minecraft might have been written in C\#. Oh wait, Microsoft already did that.
\item[{{\bf Microsoft}}]\nwl for continuing to support Minecraft when Mojang was bought by them.
\item[{{\bf Youtube Minecraft channels}}]\nwl These most certainly helped bring Minecraft into prominence, and must surely have had a hand in making Minecraft the most wanted game for ages.
\item[{{\bf Minecraft players and server owners}}]\nwl These are the real reason Minecraft has remained popular. Yes, it's had several changes and additions, multiple platforms has become a thing, and Youtube channels devoted to Minecraft have both come and gone. But without the users, Minecraft would have disappeared into the Mysts of time.

%# vim:expandtab:ts=4
\end{TPlist}
%
% end of input file: minecrafty.6
%--------------------------------------------------
\end{document}
