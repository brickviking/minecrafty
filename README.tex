\documentclass[10pt,a4paper]{book}
\author{brickviking}
\title{Minecraft}
\begin{document}
\chapter*{Minecraft}

•This is the one-stop script for starting Minecraft. This isn't a replacement for current
launchers, it's more a tool to help out with starting them up.

\section*{Features}
\begin{itemize}
\item Download commonly available launchers:
{\begin{itemize}
  \item Default java launcher for Linux (minecraft.net)
  \item ATL
  \item FeedTheBeast
  \item MagicLauncher
  \item MultiMC
  \item Technic
\end{itemize}}
\item Download available release and snapshot versions of the Minecraft java client
\item Execute any installed Minecraft java-powered launcher, even the default one
{\begin{itemize}
  \item Allocate memory for the launcher
  \item Change screen brightness/gamma/resolution
  \item Send output to a logfile 
\end{itemize}}
\item Reset refresh rate on monitors after Minecraft finishes
\item Starts a selected installed server
\end{itemize}

\subsection*{Recently removed launchers}
These launchers have all been removed from minecrafty because they're either
obsolete or in some cases, have been removed by their authors.
\begin{itemize}
  \item Aether Launcher (superceded by Aether development version)
  \item Craftland (unknown)
  \item Digiex (obsolete and not kept up to date)
  \item Minecraft Version Changer (TunkDesign) (obsolete, though still available)
  \item SKMCLauncher (sk89q.com, obsolete)
  \item Download of the latest bukkit Recommended Build (DCMA issues)
\end{itemize}

\section*{Requirements}
\begin{itemize}
  \item A paid account for Minecraft. Unpaid won't work, or at best you will have demo.
  \item One or more of the Minecraft launchers referred to in the list above.
  \item Linux running in graphical (Xorg) mode with 3-D acceleration. Minecraft won't work otherwise.
  \item Any version of Linux with bash 4 (FreeBSD may work, but only if you have bash 4 installed.) This has NOT been tested with cygwin, as Windows has its own launcher.
  \item xrandr (available in most Linuxes.)
  \item awk (gawk should be available in Linux by default.)
  \item wget (to fetch files.)
  \item Java 8 (either OpenJDK, Oracle JDK or equivalents.)
  \item Other Minecraft requirements (decent video card, recent machine, enough memory.)
  \item Some terminal, to run this script and see its output. Not absolutely mandatory, but you'll thank me if something goes wrong and you can see it.
\end{itemize}

\section*{Who this is for}
\begin{itemize}
\item People who use modified jars.
\item People who use non-standard launchers like FTB or Technic.
\item People who wish to tweak how Minecraft starts up.
\item Dual/multi-monitor setups where the second screen gets its refresh rate reset every time
  Minecraft quits.
\item People who want a single file to start whatever variant of Minecraft they have,
  whether client or server.
\item People who are reasonably comfortable with using a commandline. This program
  works well enough from .desktop files, but then you don't see any text information.
\end{itemize}

\section*{Who this is NOT for}
\begin{itemize}
\item People with no computer experience, who just want to run Minecraft without
  any changes, additions or hassles. If the base launcher does what you want, then
  this script is not for you.
\item People using the new-format java-less launcher. These usually have a
  version number of 2.00.1009 or greater. Minecrafty won't start this one without
  some help, and at least on Linux, the new launcher isn't officially supported.
\item People trying to run Minecraft without having paid for it. This script
  will not help you. Minecraft is not overly expensive in comparison to some
  other games, and is most definitely worth it.
\end{itemize}
\end{document}

